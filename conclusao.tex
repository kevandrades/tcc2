
A variável "Formato", que já era de observância do Conselho Nacional de Justiça no contexto das métricas temporais desde a instituição do PJe \cite{pje}, se mostrou relevante para a predição dos modelos, especialmente dentre os que observaram tempos médios mais altos entre o início do processo e a baixa. O estudo também mostrou que a instituição do PJe foi bem-sucedida, reduzindo drasticamente a duração média dos processos. A Justiça do Trabalho, por sua vez, está com quase todo o acervo em ambiente eletrônico. 

A variável "Grau", que já era observada como relevante para o equilíbrio do orçamento devido à disparidade de número de processos \cite{justicaemnumeros}, também se mostrou relevante para as estimativas do tempo até a baixa, aparecendo em todos os modelos, tanto quantílicos quanto gaussianos. A variável "Originário", apesar de se mostrar relevante pela diferença observada nos tempos, não entrou em nenhum modelo final, pois não trazia novas informações que já não constavam na variável "Grau", dadas as fortes associações.

Os procedimentos "Pré-processual", "Conhecimento não-criminal"  e "Outros", que estão entre os procedimentos com menores tempos até a baixa, se mostraram relevantes para boa parte dos modelos.

O número de processos na situação "Tramitando" se mostrou a variável mais relevante das variáveis quantitativas. Apesar de, individualmente, aumentar poucas horas no tempo médio de duração dos processos, ela foi a única dentre as variáveis quantitativas que apareceu consistentemente em todos os modelos, tanto quantílicos quanto gaussianos. As variáveis "Suspensos", "Despachos", "Decisões", "Conclusos", "Recurso Interno Pendente" e "Recurso Interno Julgado" também apareceram em mais de um modelo. A variável "Decisões" foi a única dentre as variáveis quantitativas citadas que apresentou influência negativa no tempo em todos eles, de modo que, quanto mais decisões forem proferidas, menor é, tanto o tempo médio, quanto o tempo para os quantis até a baixa dos processos.

O estudo demonstrou que a regressão quantílica apresenta resultados promissores no contexto da estimação de métricas temporais do Poder Judiciário, mostrando não só que essa predição é possível, como que as variáveis disponíveis publicamente são úteis para predizê-las. O método se torna mais relevante pelo fato de que os efeitos de cada variável são diferentes em cada um dos níveis de tempo.

Apesar de o estudo ser restrito à Justiça do Trabalho, modelagens semelhantes podem ser feitas para os outros ramos de justiça, sendo possível investigar a distribuição dos tempos até a baixa para cada ramo. De forma análoga, existem outras variáveis temporais oriundas do DataJud, como tempo médio do pendente líquido e tempo médio até o julgamento, que poderiam ser estudadas através de metodologias semelhantes. Outras variáveis que não estavam disponíveis no mesmo nível de agregação dos dados desse trabalho, como número de servidores e número de juízes e magistrados, permitem construir novas hipóteses e abrem espaço para estudos futuros.
\newpage