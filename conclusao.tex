
A variável Grau e a variável Formato, que já eram variáveis de observância do Conselho Nacional de Justiça no contexto das métricas temporais, como observado na implementação do PJe \cite{pje} e da Política Nacional de Priorização do Primeiro Grau \cite{justicaemnumeros} se mostraram relevantes nos modelos. Em especial, a variável Grau apareceu em todos. O estudo mostrou que a instituição do PJe foi bem-sucedida, reduzindo drasticamente a duração média dos processos. A Justiça do Trabalho, por sua vez, está com quase todo o acervo em ambiente eletrônico. 

O número de processos tramitando se mostrou a variável mais relevante das variáveis quantitativas. Apesar de, individualmente, aumentar poucas horas no tempo médio de duração dos processos, ela foi a única dentre as variáveis quantitativas que apareceu consistentemente em todos os modelos, tanto quantílicos quanto gaussianos.

Verificou-se que as decisões têm influência negativa no tempo, de modo que, quanto mais decisões forem proferidas, menor o tempo esperado até a baixa dos processos. Esse comportamento se repetiu para todos os modelos onde a variável foi incluída. 

O estudo demonstrou que a regressão quantílica apresenta resultados promissores no contexto da estimação de métricas temporais do Poder Judiciário, mostrando não só que essa predição é possível, como que as variáveis disponíveis publicamente são úteis para predizê-las. O método se torna mais relevante pelo fato de que os efeitos de cada variável são diferentes em cada um dos níveis de tempo. O modelo para a mediana, especificamente, foi útil também a nível preditivo, apresentando um erro menor que o modelo gaussiano (mesmo utilizando menos variáveis).

Embora o estudo tenha se restringido à Justiça do Trabalho, modelagens semelhantes podem ser feitas para os outros ramos de justiça, sendo possível investigar a distribuição dos tempos até a baixa para cada ramo. De forma análoga, existem outras variáveis temporais oriundas do DataJud, como tempo médio do pendente líquido e tempo médio até o julgamento, que poderiam ser estudadas através de metodologias semelhantes. Outras variáveis que não estavam disponíveis no mesmo nível de agregação dos dados desse trabalho, como número de servidores e número de juízes e magistrados, permitem construir novas hipóteses e abrem espaço para estudos futuros.
\newpage