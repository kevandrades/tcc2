O Poder Judiciário é um dos poderes fundamentais da estrutura tripartite do Estado brasileiro. Suas ações estão diretamente atreladas ao funcionamento das instituições de Estado e à manutenção da justiça, suscitando interesse para a administração pública. O entendimento da dinâmica judicial é o que garante sua transparência e credibilidade frente à sociedade civil.

Nos últimos anos, o emprego de tecnologias para aumento de eficiência dos serviços prestados tem sido uma preocupação do Poder Judiciário, e o Conselho Nacional de Justiça (CNJ) tem proposto medidas para cumprir esse fim, tais como o Processo Judicial Eletrônico \cite{pje} e o Juízo 100\% Digital \cite{juizo100digital}.

O Relatório Justiça em Números, que publica anualmente estatísticas sobre o Poder Judiciário, mostrou que, em 2021, o Brasil detinha uma relação de 8,5 magistrados para cada 100.000 habitantes, que configura aproximadamente metade do valor relativo na comparação com países europeus \cite{justicaemnumeros}. \citeonline{satiro2021desempenho}, ao investigar determinantes da produtividade do Poder Judiciário, concluíram que o número de servidores, advogados, empregados terceirizados e a carga de trabalho têm influência direta na produtividade de um tribunal.

Para este projeto, a principal variável analisada será o tempo compreendido entre o início do processo e sua baixa, que pertence à classe de indicadores de tempo de tramitação. As baixas, por sua vez, são definidas como quaisquer processos que sejam “[...] a) remetidos para outros órgãos judiciais competentes, desde que vinculados a tribunais diferentes; b) remetidos para as instâncias superiores ou inferiores; c) arquivados definitivamente.” \cite{painelestatistica}.

A relevância dessa classe de indicadores se dá pelo fato de que uma baixa configura um encerramento da situação de pendente de um processo. O estudo visa avaliar como esse tempo se distribui entre os órgãos julgadores da Justiça do Trabalho, e quais variáveis podem influenciar em sua redução ou aumento. Tal propósito será cumprido com a construção de um modelo de regressão quantílica. O modelo vai permitir observar quais varas estão abaixo ou acima de determinada faixa, analisar tendências e extrair resultados probabilísticos.



%\citeonline{david2000applied}

